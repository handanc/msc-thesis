Glyceraldehyde-3-phosphate dehydrogenase (GAPDH)

ethanolb2 24.128  2
ethanolb8 28.308  3
caffeine  48.769  1
conifel.  36.816  2
iron      25.341  3
nickel    32.787  2
phenyl.   32.740  3
silver    37.397  2
wildtype  32.867  2

Glyceraldehyde-3-phosphate dehydrogenase (GAPDH) is one of the most commonly used housekeeping genes used in comparisons of gene expression data.




Boucherié H, et al. (1995) Differential synthesis of glyceraldehyde-3-phosphate dehydrogenase polypeptides in stressed yeast cells. FEMS Microbiol Lett 125(2-3):127-33
 -------------------------------------------
Three unlinked genes, TDH1, TDH2 and TDH3, encode the glycolytic enzyme glyceraldehyde-3-phosphate dehydrogenase (triose-phosphate dehydrogenase; TDH) in the yeast Saccharomyces cerevisiae. We demonstrate that the synthesis of the three encoded TDH polypeptides (TDHa, TDHb and TDHc, respectively) is not co-ordinately regulated and that TDHa is only synthesised as cells enter stationary phase, due to glucose starvation, or in heat-shocked cells. Furthermore, the synthesis of TDHb, but not TDHc, is strongly repressed by a heat shock. Hence, the TDHa enzyme may play a cellular role, distinct from glycolysis, that is required by stressed cells.


Delgado ML, et al. (2001) The glyceraldehyde-3-phosphate dehydrogenase polypeptides encoded by the Saccharomyces cerevisiae TDH1, TDH2 and TDH3 genes are also cell wall proteins. Microbiology 147(Pt 2):411-417
-------------------------------------------
The authors show that the glycolytic enzyme glyceraldehyde-3-phosphate dehydrogenase (GAPDH) of Saccharomyces cerevisiae, previously thought to be restricted to the cell interior, is also present in the cell wall. GAPDH activity, proportional to cell number and time of incubation, was detected in intact wild-type yeast cells. Intact cells of yeast strains containing insertion mutations in each of the three structural TDH genes (tdh1, tdh2 and tdh3) and double mutants (tdh1 tdh2 and tdh1 tdh3) also displayed a cell-wall-associated GAPDH activity, in the range of parental wild-type cells, although with significant differences among strains. A cell wall location of GAPDH was further confirmed in wild-type and tdh mutants by indirect immunofluorescence and flow cytometry analysis with a polyclonal antibody against S. cerevisiae GAPDH. By immunoelectron microscopy, the GAPDH protein was detected at the outer surface of the cell wall of wild-type cells, as well as in the cytoplasm. Western immunoblot analysis of cell wall extracts and cytosol showed that Tdh2 and Tdh3 polypeptides are present in the cell wall, as well as in the cytosol, of exponentially growing cells. Tdh1 is only detected in stationary-phase cells, again in both cytosol and cell wall extracts. The results incorporate the GAPDH of S. cerevisiae, encoded by TDH1-3, into the newly emerging family of multifunctional cell-wall-associated GAPDHs which retain their catalytic activity.


McAlister L and Holland MJ (1985) Isolation and characterization of yeast strains carrying mutations in the glyceraldehyde-3-phosphate dehydrogenase genes. J Biol Chem 260(28):15013-8
-------------------------------------------
Mutant yeast strains were constructed which carry insertion mutations in each of the glyceraldehyde-3-phosphate dehydrogenase structural genes which have been designated TDH1, TDH2, and TDH3. Haploid strains carrying mutations in TDH1 and TDH2 as well as TDH1 and TDH3 were isolated from crosses between strains carrying the appropriate single mutations. The three single mutants as well as the two double mutants grow at wild type rates when ethanol is used as carbon source. Mutant strains lacking only a functional TDH2 allele or a TDH3 allele grow at 50 and 75\% of the rate observed for wild type cells, respectively, when glucose is used as carbon source. No growth phenotype was observed for strains lacking only a functional TDH1 allele when either fermentable or nonfermentable carbon sources were used. Evidence is presented that strains lacking functional TDH2 and TDH3 alleles are not viable. These data demonstrate that the presence of a functional TDH2 or TDH3 allele is required for cell growth.








-------------------------------------------
-------------------------------------------

FROM Petek


ethanol
-------------------------------------------
Our transcriptome analysis of the ethanol-tolerant B2 and B8
clones revealed that beside a high enrichment of development and
mating category due to diploidization, the other most enriched
functional classes were comparable to those previously reported
for ethanol tolerance

Also, very common to previous works was the up-regulation of genes implicated in storage carbohydrates metabolism. However, only B2 strain exhibited higher accumulation of glycogen, indicating that the expression changes in thosegenes are not directly linked to metabolic changes

Proteomics studies revealed decreased abundance in proteins related to mitochondrial integrity in the ethanol tolerant and cross-resistant clone B8, suggesting that increased ethanol productivity might be at least partially due to reduced respiratory activity. This observation is consistent with previous reports targeting ethanol production by using respiratory-deficient yeast strains (65,66).

Higher abundance levels of ribosomal proteins, amino acid metabolism, and glycolysis in clone B8 is consistent with a higher fermentation capacity of this evolved clone as compared to reference strain, as shown in our fedbatch process fermentation study. Thus, these results support previous suggestions that increased ethanol tolerance can be accompanied by improved fermentation performance (67)


CAFFEINE
-------------------------------------------
The use of chemical mutagens such as ethyl methanesulfonate (EMS) usually results in an accelerated and efcient selection process, where higher concentrations of the stress factors could be tolerated by the evolving populations in relatively early populations of selection, compared to a selection without prior EMSmutagenesis. Additionally, the EMS-mutagenized S. cerevisiae populations usually have higher survival rates during selection, and can survive signifcantly higher stress levels by the end of the selection, compared to non-mutagenized populations (Çakar et al. unpublished data). Surprisingly, during our cafeine stress selection, the populations obtained without prior EMS mutagenesis could survive very high cafeine levels (up to 50 mM for the fnal population with a survival rate of about 60\%), and this performance was almost the same as that of the EMS-mutagenized population of cafeine selection (data not shown).

The inhibitory efect of cafeine on the growth of various organisms such as bacteria, yeasts, plants as well as human cancer cell lines is well-known. Our cafeine-hyperresistant mutant strain Caf905-2, however, did not show any decrease in µmax. Additionally, Caf905-2 could grow at cafeine levels as high as 50 mM There are some reports in the literature about the improvement of cafeine resistance in yeast up to about 20 mM cafeine by mutation or overexpression of various genes, such as TOR1 (Reinke et al. 2006), RHO5 (Schmitz et al. 2002), SIT4 (Hood-DeGrenier 2011). However, to our knowledge, there is no previous report on any mutant yeast strain that could resist as high as 50 mM caffeine stress.

Our comparative genome analysis results revealed single nucleotide polymorphisms (SNPs) in only three genes of the cafeine-resistant evolved strain: PDR1, PDR5 and RIM8

Among those, CYC7 was the most upregulated gene in Caf905-2 by about 222-fold, and was placed in the category of electron transport and membrane-associated energy conservation (Supplementary Table 1). CYC7 is a hypoxic gene that is transcribed optimally under anoxic or microaerophilic conditions. It is not expressed until the oxygen concentration is below 0.5 mM O2 (Burke et al. 1997) The exposure of S. cerevisiae to antimycin A, a respiratory chain complex III inhibitor, also enhanced the expression of the hypoxic gene CYC7 (Liu and Barrientos 2013). Considering that Caf905-2 was highly cross-resistant against antimycin A stress (Fig. 3), the signifcantly upregulated hypoxic genes CYC7 may play a critical role in cafeine and antimycin A resistance by their efects on mitochondrial energy metabolism. It is also important to note that PDR1 controls many genes placed in the ‘Metabolism of energy reserves’ category in our transcriptomic data

These results are in line with our previous reports on other S. cerevisiae mutants that are resistant to coniferyl aldehyde (Hacısalihoğlu et al. 2019), ethanol (Turanlı-Yıldız et al. 2017), and nickel stress (Küçükgöze et al. 2013), and also the chronologically long-lived S. cerevisiae mutant (Arslan et al. 2018), in which many of these trehalose- and glycogen-related genes were also upregulated.

Additionally, it has been previously reported that cafeine and rapamycin commonly induce genes involved in glycogen and trehalose metabolism (Kuranda et al. 2006; Hardwick et al. 1999).

Considering the fact that Caf905-2 was cross-resistant to coniferyl aldehyde and propolis stress (Fig. 3), and a coniferyl aldehyde-resistant S. cerevisiae strain was also highly cross-resistant to cafeine stress, along with an upregulated SNQ2 gene (Hacısalihoğlu et al. 2019); SNQ2 could thus be considered as a common factor that may be involved in resistance to cafeine, propolis and coniferyl aldehyde


 coniferylaldehyde
-------------------------------------------

Although there is a previous report on the transcriptomic changes of a commercial
baker‘s yeast strain upon exposure to CA (Sundström et al. 2010), to our knowledge, there are
no reports on the transcriptomic changes in a CA-resistant yeast strain. Our microarray results
revealed that a number of transcripts encoding NAD(P)-dependent aldehyde dehydrogenases
were differentially regulated in the CA-resistant strain even without CA treatment.

Among these, all members of the ALD gene family except for ALD5 were upregulated. Aldehyde
dehydrogenases and decarboxylases have been suggested to have an active role in CA
conversion in S. cerevisiae (Adeboye et al. 2017). In addition, the enhanced vanillin
conversion rate of a vanillin-tolerant S. cerevisiae strain was associated with the upregulation
of several oxidoreductases (Shen et al. 2014).

Similar to that vanillin-tolerant strain, BDH2,
YPL113C and YJR096W genes encoding oxidoreductases were upregulated in our CAresistant strain. In addition to these, putative aryl-alcohol dehydrogenases YPL088W and
AAD15 were also induced in our CA-resistant strain (Table 2). However, the role of these
dehydrogenases in CA conversion has not been studied yet. Thus, these genes could be
potential candidates for CA conversion and/or resistance.

Phenolic compounds including ferulic acid, vanillin and CA induce accumulation of
ROS (Nguyen et al. 2014, Fletcher et al. 2018). In our CA-resistant strain, transcription levels
of the genes involved in oxidative stress response and oxidant detoxification were higher than
in the reference strain. Overall, there were 11 upregulated antioxidant genes, including thioredoxin peroxidase Tsa2p and sulfiredoxin Srx1p, annotated to an antioxidant-related GO
term (Table 2). Therefore, the CA-resistant strain seems to have a stimulated antioxidant
defense system. Such defense systems are crucial in the maintenance of cellular redox status
and they protect cells against the detrimental effects of ROS (Morano et al. 2012). Formation
of stress granules (SG) was also reported in yeast cells treated with vanillin (Iwaki et al.
2013).

Transcriptional reconfiguration of the genes involved in maintaining energy and redox
balance has been associated with tolerance to furfural, HMF (Liu et al. 2009; Ask et al. 2013)
and ethanol (Stanley et al. 2010) in S. cerevisiae. In the CA-resistant strain, even in the absence of CA stress, glucose uptake and metabolism were enhanced, as indicated by the
upregulation of the genes encoding high affinity hexose transporters (HXT6, HXT7 and HXT8)
and the enzymes involved in several intermediate steps of glycolysis (HXK1, GLK1, TDH1,
GPM2, ERR1 and PYK2). Further, NAD(P)H regeneration appeared to be upregulated by
TDH1 in glycolysis, GND2 in pentose phosphate pathway and GCY1 in glycerol metabolism

In addition, the genes involved in the biosynthesis of storage carbohydrates were found to be
upregulated. In line with that, the CA-resistant strain accumulated higher levels of both
glycogen and trehalose (Fig. 6), even in the absence of any CA stress, implying that it is
―ready‖ for CA stress even at non-stress conditions. Additionally, the cellular trehalose
dynamics in response to CA stress was also different between the CA-resistant BH13 and the
reference strain, possibly because the perceived level of stress by the two strains was not the
same. This might be partly due to the enhanced CA conversion in BH13 that results in a rapid
decrease in CA to possibly milder stress levels.


iron
-------------------------------------------
In this study, we have obtained an iron‐resistant S. cerevisiae mutant, M8FE, which showed cross‐
resistance to other metals, such as chromium, nickel and cobalt.

Our microarray results revealed that PHO84 was the most downregulated gene in M8FE under
both iron stress and control conditions. Our current study suggests that the mutant’s iron
resistance and even its cross‐resistance to other transition metals might be related to the down‐
regulation of PHO84, according to the close connection between intracellular phosphate and iron
homeostasis.

According to the microarray results, many genes involved in oxidative stress response were
upregulated in M8FE under control conditions. As the iron content of the mutant was also very high, this would lead to oxidative stress or damage in the cell, as reported previously [59,60]]. However, intracellular ROS amounts of the mutant were lower than those of the reference strain, implying that the upregulated oxidative stress‐related genes in M8FE most likely help reduce the intracellular oxidative levels.

It is known that the “ribosome biogenesis” process requires high cellular energy [72]. It can therefore be suggested that the mutant prefers to save its energy as trehalose, rather than consumingmit in ribosome biogenesis.

Our transcriptomic analysis results revealed that the gene GPH1, responsible for glycogen degradation, was also upregulated by 17.2‐fold. The glycogen degradation product “Glucose 1‐P” can be used as a substrate for trehalose biosynthesis or it can be converted to “Glucose 6‐P” by the help of phosphoglucomutase encoded by PGM2 (19.6‐fold upregulated in M8FE). Glucose 6‐P would then be either used in glycolysis or glycerol pathways. Glycerol is produced by S. cerevisiae to cope with the osmotic stress, to manage cytosolic phosphate levels or to maintain NAD+/NADH redox balance [73]

The genes GPD1 and GPD2, which are responsible for the synthesis of glycerol 3‐phophate
dehydrogenase, were also upregulated in M8FE under control conditions. These results also support
the increased glycerol biosynthesis observed in M8FE. To our knowledge, there have been no
previous reports implying a relationship between metal stress and glycerol production.


nickel
-------------------------------------------
Exposure of S. cerevisiae cells grown in YPD medium to a pulse stress of 25 mM NiCl2 followed by
DNA microarray analysis revealed that genes related to sulphur amino acid metabolism,
iron metabolism and heavy metal homeostasis were significantly upregulated.
Additionally, genes related to oxidative stress response, i.e. gluthathione peroxidase,
thioredoxin and glutaredoxin, and DNA repair were induced. Transcriptomic data were similar to those obtained previously where cobalt stress was shown to induce iron
accumulation in yeast cells (Stadler & Schweyen, 2002). Thus, it was suggested that
cellular iron is increased by the transcriptional response to nickel stress and also cobalt
stress. In the present study on nickel hyper-resistant mutant M9, however; only 4 (GTT2,
MHT1, STR2, and STR3) out of 26 upregulated genes related to sulphur amino acid
metabolism in nickel-exposed S288C (Takumi et al., 2010) were found to be also
upregulated in M9, and the level of upregulation of those genes in M9 was significantly
lower than in S288C

Additionally, genes related to DNA repair were also not
upregulated in M9. Besides; stress response genes, genes related to protein refolding,
trehalose, glucose and amino acid metabolism, glycogen biosynthesis were upregulated
in M9, but they were not found to be upregulated in the reference strain of Takumi et
al., (2010) upon nickel exposure.  The upregulated gene groups that are common to both
M9 and the reference strain in Takumi’s work are those related to iron metabolism,
metal homeostasis, cellular response to oxidative stress, and xenobiotics resistance
(GTT1 and GTT2) (Table 2; and Takumi et al., 2010).

Detailed analysis of
both up- and down-regulated genes in M9, however, revealed that the genes related to
cell rescue, defense, virulence, subcellular localization, metabolism and cellular
transport were upregulated, unlike the reference strain in Takumi et al., (2010).


silver
-------------------------------------------
Niazi et al. (2011) showed that silver stress caused a strong upregulation of the CUP1-1 and CUP1-2 genes encoding a metallothionein protein that binds copper and confers high copper resistance in S. cerevisiae.The cross-resistance analysis results of our silver-resistant strain 2E showed that it was also hyper-resistant to copper stress . Additionally, genes encoding the copper-binding proteins CUP1-1, CUP1-2, CTR3, and the transcription factor CUP2 were also upregulated in 2E, compared to the reference strain.

It has been shown that silver stress results in a decrease in the mitochondrial activities, such as cellular respiration, and an increase in oxidative stress (Horstmann et al., 2019; Marquez et al., 2018). It has been recently reported that the exposure of wild-type S. cerevisiae cells to silver stress resulted in the differential expression of some oxidative stress responsive genes: for example, LTV1 was upregulated, and TSA2 and ZTA1 were downregulated (Horstmann et al., 2019). However, in our silver-resistant evolved strain 2E that was highly cross-resistant to oxidative stress, LTV1 was 2.75-fold downregulated, TSA2 and ZTA1 were 6.95-fold and 7.98-fold upregulated, respectively.

The evolved strain 2E also had a different mitochondrial gene expression profile,
compared to the reference strain: 100 mitochondrial genes were downregulated, and 210 were
upregulated in 2E. The upregulated ones involve proteins participating in the electron transfer system, such as CYC7, COX7, NDI1, SDH1, COX13, SDH4, YTP1, RIP1, SDH3, and SDH2, indicating a more active aerobic metabolism. In 2E, the gene encoding oxidative-stress
related glutathione peroxidase 1 (GPX1) was also upregulated, compared to the reference
strain, along with many genes in the oxidoreductase activity category (Table 1). GRX1 and
GRX2 genes were also upregulated in 2E, compared to the reference strain, along with the
SOD1 and SOD2 genes encoding cytosolic and mitochondrial superoxide dismutases (Table
1).  It has been shown that the deletion mutants of S. cerevisiae lacking the genes SOD1 and ETR1 (the gene encoding 2-enoyl thioester reductase) were sensitive to silver stress (Marquez et al., 2018). In our evolved strain 2E, both of these genes were upregulated by 2.29 and 6.69-fold, respectively, implying the potential role of these two genes in resistance to silver and oxidative stress. ...  These studies imply that silver leads to oxidative damage in cells, and might help explain the oxidative stress resistance of the mutant 2E.

Their transcriptomic analysis results revealed that many genes implicated in cell wall organization and biogenesis were differentially expressed: RNT1 and YVH1 were upregulated, and genes encoding cell wall mannoproteins such as TIP1, TIR1, TIR2, TIR3, TIR4 and DAN1 were downregulated. TIP1 and TIR1-4 genes were also downregulated in our silver-resistant strain 2E. Horstmann et al. (2019) attributed the significant downregulation of cell wallrelated genes to the possibility that the cells may sense their cell wall as abnormally thick because of the AgNPs. The accumulation of the related cell wall mannoprotein and glucans may result in the downregulation of the related genes. They also indicated that genes for membrane sugar transporters (such as HXT13, HXT17, and HXT2), ergosterol synthesis (e.g. ERG11, 25, 28,3,5 and 6), tricarboxylic acid (TCA) cycle (ACO1, CIT2, CIT3) and NADH regeneration (ALD4 and GUT2) were downregulated. Additionally, their cell wall stability assay results showed that the treatment with AgNPs made the yeast cells more susceptible to cell wall damage, compared to the control cells

Our results revealed that genes involved in cell wall/membrane integrity,
endocytosis and vesicular transport activities, oxidative metabolism, cellular respiration and
copper homeostasis may play a key role in silver resistance of S. cerevisiae.
