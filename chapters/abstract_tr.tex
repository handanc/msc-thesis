Bir organizmanın çeşitli stres koşullarına tolerans düzeyini artırmak, genellikle analiz etmek için disiplinler arası araştırma gerektiren yüksek miktarda genomik ve deneysel veri sağlayan adaptif laboratuvar evrimi (ALE) deneylerinin en yaygın uygulamalarından biridir.

Bu çalışmada, etanol, kafein, koniferilaldehit, demir, nikel, feniletanol ve gümüş stresi koşullarında evrimleşen \emph{S. cerevisiae} suşlarının ekspresyon profilleri ve enzim kinetiği, evrimleşmiş suşların davranışlarını simüle etmek için genom ölçekli metabolik model ile bütünleştirilmiştir. Akı denge ve değişkenlik analizi, dayanıklılık analizi, duyarlılık, hassaslık ve rastgele örneklendirme gibi geleneksel metotlar, hücresel metabolizmadaki en ortak ve farklılaşmış noktaları belirlemek için uygulanmıştır. Yeniden yapılandırılmış modeller, stresli ortamlarda hayatta kalabilmek için ana hedefini verimli protein pay ayırma olarak düzenlemiş hücrelerin deney koşullarını simüle edebilmiştir. Simülasyon sonuçları, hem küçük hem de büyük ölçekli olarak, hem bireysel hem de karşılaştırmalı olarak incelenmiş, adaptasyon mekaniğinden sorumlu olduğu öngörülen reaksiyonlar, metabolitler ve/veya enzimlerden en ortak ve değişken davranış gösterenleri mutant suşlar arasında tanımlanmış vetartışılmıştır.

Yapılan çalışmalar, genel maya metabolizması hakkında sistematik bir bakış açısı sağlarken, genom ölçeğinde metabolik modelleme tekniklerini ALE deneylerinin verileri üzerinde başarıyla kullanmıştır.
