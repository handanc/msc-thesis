
\chapter{CONCLUSION}

In this study, constraint-based genome-scale metabolic modelling techniques are used to analyze multiple mutant strains that have gained resistance against different stress conditions by adaptive laboratory evolution. Total of 9 different metabolic models for the evolved ethanol, caffeine, coniferylaldehyde, iron, nickel, phenylethanol, and silver tolerant strains, and the reference strain are reconstructed from the consensus \emph{Saccharomyces cerevisiae} model Yeast8. In the reconstruction process, reactions are enzymatically constrained and the differential expressional profiles are integrated to gather more biologically relevant results. FBA, FVA, MOMA, PhpPP, robustness, survivability and random sampling analyses are conducted in computational environment for each mutant strain, and the results are compared both individually and comparatively.

Despite the lack of metabolomics and fluxomics data, findings of the study contributes to the research in adaptation mechanism by providing a more systematic view compared to a sole differentially expression analysis. Even with the low coverage of transcriptional integrations, multiple targets of common or divergent behaviors of mutant strains are reported and discussed.

One of the main findings of the study is that the protein allocation is highly intertwined with the energy metabolism. As the kinetic parameters integrated into metabolic models allow to follow each flux distribution without needing any \emph{ad hoc} constraints, the only difference making part of each model was the protein allocation which models decide to use preferred enzymes for the best outcome. As models used their proteins the most efficient way in terms of to fullfill objective (such as biomass production), the converging difference became the different behaviors in the energy related reactions, especially in the coenzyme (NAD, NADP, FAD) utilizing reactions, and the glutathione metabolism. Secondly, sensitivies to the exchange metabolites differed in each strain, such as oxygen, biotin, riboflavin and flavin mononucleotide. Most of the metabolites that are found as target were closely related to stress response of yeast, therefore commonalities within strains were treated as a result of the stress response, and divergent findings were considered as adaptive effects. Lastly, targets that affect the composition of the lipids, and therefore cell wall formation, was another noteworthy finding of the study since the connection between the two were previously reported multiple times.

Although our findings tried to capture the adaptive metabolism in a more systematic broader way compared to the differential gene expression analysis; the results were still too abundant to investigate each. In the near future, with more precise data made available in ALE experiments, the essense of the adaptive metabolism could easily be captured with metabolling.
