\chapter{INTRODUCTION}

\section{Metabolic Networks}
\subsection{Mathematical Modeling of Cellular Metabolism}

Mapping the network of the chemical reactions in the cells is another field of synthetic biology. Organism specific reconstructed metabolic networks allow researchers to design experiments, and even to predict outcomes before the experimental works. These networks may also be used to build mathematical models which can simulate metabolic fluxes reflecting the reality (Thiele I, Palsson BO., 2010).

One of the most extensively studied metabolic networks is the model of S. cerevisiae. More than 25 models of metabolic network for the yeast have been published since 2003 (B.D. Heavner, N. D. Price, 2015), and the number is increasing. Researchers have combined the experimental data into models to simulate phenotypes in different environmental conditions (Snitkin et al., 2008).


Indeed, the ultimate goal of many fields of the life sciences is to understand cell physiology and hence the flux distribution, which characterizes the metabolic state of a cell. Based on this information, it will be possible to design strategies, for example, to avoid certain phenotypes (applications in infection biology and cancer research) or to redirect fluxes (applications in industrial biotechnology to increase rate, yield and titre of a product of interest). \cite{blank2017let}

\subsection{Steady State Assumption}

A rather frequently used assumption for metabolic network modelling is that the production and consumption of internal metabolites must balance (steady-state assumption). \cite{reimers2016steady}


\subsection{Genome Scale Metabolic Models}
\subsection{Constraint-Based Models}
\subsection{Flux Balance Analysis}

\section{\emph{Saccharomyces cerevisiae}}
\subsection{Industrial importance of \emph{S. cerevisiae}}

\emph{Saccharomyces cerevisiae} is one of the main microorganisms used in the biochemical industry such as alcohol fermentations, baking processes and bio-ethanol production


\subsection{Metabolic Models of \emph{S. cerevisiae}}
\subsection{Applications of \emph{S. cerevisiae} GSMMs}

Citation example is \cite{aran2006signlanguagetutoring}.
