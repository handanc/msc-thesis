\chapter{FLUX BALANCE ANALYSIS}

\section{Introduction}

\subsection{Understanding control in metabolic models}

Early enzymology assumed the existence of rate limiting steps in biological pathways. The intuition was that the overall rate of a pathway is constrained by the rate of the slowest step. As the slowest reaction rate increases the whole pathway rate increases in proportion until another step becomes limiting. Niederberger \emph{et. al} tested this assumption and found in contrast that individual up or down regulation of enzyme quantities at specific reaction steps had only marginal effect on overall tryptophan biosynthesis pathway in \emph{Saccharomyces cerevisiae}. The rate of the pathway was instead accelerated by increasing the quantity of five related enzymes in tandem. This research demonstrated control in a biological system is distributed over the system as a whole rather than concentrated at individual reactions.

The theory of metabolic control analysis (MCA)  states that there is no rate limiting step in a pathway, but instead each reaction shares a measure of overall control. The control the rate of reaction
