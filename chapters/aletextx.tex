\cite{mcdonald2019microbial}

Microbial experimental evolution uses controlled laboratory populations to study the mechanisms of evolution. The molecular analysis of evolved populations enables empirical tests that can confirm the predictions of evolutionary theory, but can also lead to surprising discoveries. As with other fields in the life sciences, microbial experimental evolution has become a tool, deployed as part of the suite of techniques available to the molecular biologist.



\cite{garland2009experimental}
What is experimental evolution? We use the term to mean research in which populations are studied across multiple generations under defined and reproducible conditions, whether in the laboratory or in nature (for recent overviews, see Bennett 2003; Garland 2003; Swallow and Garland 2005; Chippindale 2006; Garland and Kelly 2006). This intentionaly general definition subsumes various types of experiments that involve evolutionary (cross-generational, genetically based) changes. At one end of the continuum, the study of evolutionary responses to naturally occurring events (e.g., droughts, fires,invasions, epidemics) may constitute a kind of adventitious experimental evolution, especially if these events occur repeatedly and predictably enough that the study can be replicated, either simultaneously or in subsequent years.

“Laboratory natural selection” denotes experiments in which the environment of a laboratory-maintained population is altered (e.g., change of temperature, culture medium, food) as compared with an unaltered control population. “Laboratory culling” involves exposing an experimental population to a stress that is lethal (or sublethal) and then allowing the survivors (or the hardiest) to become the parents of the next generation. In all of the foregoing types of experiments, the investigator does not specifically measure and select individuals based on a particular phenotypic trait or combination of traits. Rather, selection is imposed in a general way, and the population has relatively great freedom to respond across multiple levels of biological organization (e.g., via behavior, morphology, physiology). “Multiple solutions” (different adaptive responses among replicate lines) are possible and even probable, depending on the kind of organism and experimental design.

In classical “artificial selection” or “selective breeding” experiments, individuals within a population are scored for one or more specific traits, and then breeders are chosen based on their score (e.g., highest or lowest). Depending on the level of biological organization at which selection is imposed—and the precision with which the phenotype is defined in practice—multiple solutions may again be common (Garland 2003; Swallow et al. this volume).

Domestication is an interesting (and ancient) type of experimental evolution that generally involves some amount of intentional selective breeding.

What we are terming “experimental evolution” clearly covers a broad range of possible experiments. Historically and at present, different methodologies for experimental evolution have been and are being applied unequally across levels of biological organization (e.g., behavior, life history, physiology, morphology) and across kinds of organisms (e.g., bacteria, Drosophila, rodents).

In any case, to qualify as experimental evolution, we require most if not all of the following fundamental design elements: maintenance of control populations, simultaneous replication, observation over multiple generations, and the prospect of detailed genetic analysis. In short, experimental evolution is evolutionary biology in its most
empirical guise.

macroevolution is generally used to refer to change at or above the level of the species, including long-term trends and biases that are observed in the fossil record. Macroevolutionary phenomena are difficult to study experimentally because of the long time scales involved. One consequence of this is that many creationists accept microevolution as fact—how could they not if they drink cow’s milk from a modern dairy or eat sweet corn?—but reject the fact
of macroevolution.


\cite{loewe2009framework}
Mutations with weak effects on fitness that interact with each other are of great interest to evolutionary genetics and genomics, as their long-term consequences are much harder to predict than those of mutations with large effects. These mutations with small effects are also much more frequent


\cite{dujon2010yeast}
Yeasts offer unique advantages for evolutionary genomic studies among eukaryotic organisms. A major, and unexpected, lesson from yeast genomics is the extensive sequence divergence observed between different lineages. This divergence goes right down to the species level and reflects intense genomic changes that contrast with the conservation of biological properties of yeasts for very long evolutionary times. Rather than offering a continuous range of gradual evolutionary adaptations, as proposed by classical Darwinian theory, genomes from distinct yeast clades, or even from species of the same genus, differ from one another in an abrupt manner. This is consistent with yeast genomes being the remnants of repeated bottleneck events that occurred in essentially clonal populations. The stochastic drift resulting from such a mode of propagation is important as it offers the possibility for non-optimized variants to survive and eventually colonize novel niches to which they may be better adapted. The recurrent emergence of this unicellular mode of life — with its specific reproductive properties — during fungal evolution is one of the main interests for evolutionary scientists studying yeasts, as is the conservation of this mode over long evolutionary periods. Now that population structures of yeasts can be readily studied at the genomic level, this opens a new perspective for evolutionary genomics. The flow of genetic material within and between populations or even species can now be precisely measured and its consequences carefully examined.




From the perspective of human health, a major goal of genomics work is to understand not only the function of human genes, but also the impact of mutations in those genes, and how drugs can be designed to modify or repair those functions. Yet we humans are neither sufficiently genetically variable nor amenable to experimentation that the function of most genes could be ascertained from just our species. An immeasurable benefit to understanding human genetics comes from work on other species—model organisms. As we now know, work on the genetics of other eukaryotes can often be extrapolated to humans. One recent example is the identification of a human gene responsible for a sleep disorder based on the Drosophila circadian clock gene per (Toh et al. 2001).




Serial batch transfer and continuous bioreactors have been commonly used for evolution experiments. Serial batch transfer has the advantage of being simple to set up and utilize but are more susceptible to random drift due to repeated population bottlenecks. Continuous systems, though requiring a greater initial investment in equipment, avoid this downside by obviating the need for repeated inoculations. The relative merits of these distinct experimental setups for different situations are debated below. In addition to how the experiments are carried out, the strain of interest is also a factor. Though any culturable microbial strain can be used in laboratory evolution, the traditional workhorses (Escherichia coli and Saccharomyces cerevisae) are most often used .
