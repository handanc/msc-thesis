Increasing an organisms’ tolerance level to various stress conditions is one of the most common applications of adaptive laboratory evolution (ALE) experiments which provides high amounts of genomic and experimental data that usually requires interdisciplinary research to analyze.

This study integrates enzyme kinetics and expression profiles of multiple \emph{S. cerevisiae} strains that have evolved in 8 stress conditions, ethanol, caffeine, coniferylaldehyde, iron, nickel, phenylethanol, and silver, into the genome-scale metabolic model of yeast to simulate evolved strains’ behaviors. Traditional analyses such as FBA, FVA, MOMA, PhPP, robustness, survivability, sensitivity, and random sampling are conducted to identify the most common and divergent points in cellular metabolism. Reconstructed models were able to simulate batch conditions where efficient protein allocation is the main goal for cells to survive under stressful environments. Simulation results were investigated both individually and comparatively, on both small-scale and large scale. The most common and varying targets of reactions, metabolites, and enzymes that are assumed to be responsible for the adaptation mechanics were identified and discussed across mutant strains.

The study presented here successfully used metabolic modeling techniques on ALE data while providing a systematic view of the general yeast metabolism.
